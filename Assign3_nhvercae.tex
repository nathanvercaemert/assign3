% Created 2022-04-06 Wed 08:13
% Intended LaTeX compiler: pdflatex
\documentclass[11pt]{article}
\usepackage[utf8]{inputenc}
\usepackage[T1]{fontenc}
\usepackage{graphicx}
\usepackage{longtable}
\usepackage{wrapfig}
\usepackage{rotating}
\usepackage[normalem]{ulem}
\usepackage{amsmath}
\usepackage{amssymb}
\usepackage{capt-of}
\usepackage{hyperref}
\usepackage{minted}
\usepackage{color}
\usepackage{listings}
\usepackage{placeins}
\author{Nathan Vercaemert}
\date{2022-04-05}
\title{Assignment 3\\\medskip
\large CSC 520 Spring 2022 001}
\hypersetup{
 pdfauthor={Nathan Vercaemert},
 pdftitle={Assignment 3},
 pdfkeywords={},
 pdfsubject={},
 pdfcreator={Emacs 27.2 (Org mode 9.5.1)}, 
 pdflang={English}}
\begin{document}

\maketitle
\tableofcontents

\section{Question 1}
\label{sec:org529c660}
\subsection{a.}
\label{sec:orgb81a4e4}
\begin{enumerate}
\item \(tea(BlackTea) \land (\forall x \:cheese(x) \Rightarrow \lnot tea(x))\)
\item \(blend(BlackTea, GreenTea)\)
\item \(\forall x,y \: (\lnot tea(x) \land tea(y)) \Rightarrow \lnot blend(y,x)\)
\item \(\exists x \: tea(x) \land unoxidize(x) \land \lnot wilted(x)\)
\item \(\exists x \forall y \: tea(x) \land ((tea(y) \land \lnot unoxidize(y)) \Rightarrow blend(x,y))\)
\end{enumerate}
\subsection{b.}
\label{sec:org36c77c6}
\subsubsection{1.}
\label{sec:orgc3a5b7e}
\begin{itemize}
\item \(tea(BlackTea) \land (cheese(x) \Rightarrow \lnot tea(x))\)
\item \(tea(BlackTea) \land (\lnot cheese(x) \lor \lnot tea(x))\)
\end{itemize}
\subsubsection{2.}
\label{sec:orgb386571}
\begin{itemize}
\item \(blend(BlackTea, GreenTea)\)
\end{itemize}
\subsubsection{3.}
\label{sec:orgf67b05a}
\begin{itemize}
\item \((\lnot tea(x) \land tea(y)) \Rightarrow \lnot blend(y,x)\)
\item \(\lnot(\lnot tea(x) \land tea(y)) \lor \lnot blend(y,x)\)
\item \((tea(x) \lor \lnot tea(y)) \lor \lnot blend(y,x)\)
\item \(tea(x) \lor \lnot tea(y) \lor \lnot blend(y,x)\)
\end{itemize}
\subsubsection{4.}
\label{sec:org1491708}
\begin{itemize}
\item \(tea(T1) \land unoxidize(T1) \land \lnot wilted(T1)\)
\end{itemize}
\subsubsection{5.}
\label{sec:org4751bf5}
\begin{itemize}
\item \(tea(T2) \land ((tea(y) \land \lnot unoxidize(y)) \Rightarrow blend(T2,y))\)
\item \(tea(T2) \land (\lnot(tea(y) \land \lnot unoxidize(y)) \lor blend(T2,y))\)
\item \(tea(T2) \land ((\lnot tea(y) \lor unoxidize(y)) \lor blend(T2,y))\)
\item \(tea(T2) \land (\lnot tea(y) \lor unoxidize(y) \lor blend(T2,y))\)
\end{itemize}
\subsection{c.}
\label{sec:org951c55f}
In order to prove by resolution, we introduce the following to our KB:
\begin{enumerate}
\item \(blend(BlackTea,cheese(x))\)
\end{enumerate}
note that there is implied universal quantification
\subsubsection{Also in our knowledge base}
\label{sec:orgd2b69dc}
\begin{itemize}
\item 2. \(tea(BlackTea)\)
\item 3. \(\lnot tea(cheese(x))\)
\end{itemize}
\subsubsection{Standardized for resolution}
\label{sec:org9745b2c}
\begin{itemize}
\item 4. \(tea(z) \lor \lnot tea(y) \lor \lnot blend(y,z)\)
\end{itemize}
\subsubsection{Resolution}
\label{sec:org042e888}
\begin{itemize}
\item 3. \(tea(z) \lor \lnot tea(y) \lor \lnot blend(y,z)\) resolves with 2. \(tea(BlackTea)\)
\item this leaves 4. \(tea(z) \lor \lnot blend(BlackTea,z)\) after the substitution \({y/BlackTea}\)
\item 4. \(tea(z) \lor \lnot blend(BlackTea,z)\) resolves with 3. \(\lnot tea(cheese(x))\)
\item this leaves 5. \(\lnot blend(BlackTea,cheese(x))\) after the substitution \({z/cheese(x)}\)
\item 5. \(\lnot blend(BlackTea,cheese(x))\) resolves with 1. \(blend(BlackTea,cheese(x))\)
\item this leaves the empty set
\item this proves that cheese does not blend with BlackTea.
\end{itemize}
\begin{enumerate}
\item Note
\label{sec:orgac3fbfc}
\(cheese(x)\) is being used in place of any instatiation of x that would be true in our world. It is implied that we have at least one object \(C\) that maps to a cheese, so where I use \(cheese(x), cheese(C)\) could be substituted. This could equivalently simply be \(C\) as the problem's use of "cheese" could simply be a constant. The problem's use of "cheese" is not clear in this respect. The resolution works conceptually either way. Similarly a Skolem function could be used, cheese(C(x)). If it is required that I pick one, simply "C" is the most clear because Sentence 1 in the question refers to cheese as a constant like BlackTea.
\end{enumerate}
\section{Question 2}
\label{sec:org576ed9d}
\begin{itemize}
\item Note that I use PL-FOL to refer to propositionalization.
\item Note that "mp" stands for Modus Ponens.
\end{itemize}
\subsection{a.}
\label{sec:orgec3d578}
\subsubsection{Provided Sentences in FOL}
\label{sec:org0dc4fbe}
\begin{itemize}
\item Sentence 1: \(\forall x \: o(x) \Rightarrow l(x)\)
\item Sentence 2: \(\forall x \: o(x) \Rightarrow d(x)\)
\item Sentence 3: \(\forall x \: (o(x) \land i(x)) \Rightarrow s(x)\)
\item Sentence 3 simplified: \(\forall x \: i(x) \Rightarrow s(x)\)
\item The remaining sentences will take this simplified approach. It is a safe assumption that our model does not have any assignments that would complicate these simplifications.
\item From Sentence 5: \(\forall x \: o(x) \Rightarrow (g(x) \lor si(x))\)
\item From Sentence 8: \(\forall x \: (g(x) \land i(x)) \Rightarrow pre(x)\)
\item From Sentence 8: \(\forall x \: si(x) \Rightarrow \lnot pre(x)\)
\end{itemize}
\subsubsection{Provided sentences that can be interpreted in terms of PL-FOL}
\label{sec:org1135e9c}
\begin{itemize}
\item Sentence 4: \(\lnot s(C)\)
\item Sentence 6: \(o(C)\)
\item Sentence 7: \(g(C) \land \lnot i(C) \land \lnot s(C) \land \lnot pre(C)\)
\end{itemize}
\subsubsection{FOL Sentences instantiated into PL (Carol is the only customer, so she is the only instatiation that matters)}
\label{sec:org3bd2556}
\begin{itemize}
\item Sentence 1 PL-FOL: \(o(C) \Rightarrow l(C)\)
\item Sentence 2 PL-FOL: \(o(C) \Rightarrow d(C)\)
\item Sentence 3 PL-FOL: \(i(C) \Rightarrow s(C)\)
\item Sentence 5 PL-FOL: \(o(C) \Rightarrow (g(C) \lor si(C))\)
\item From Sentence 8 PL-FOL: \((g(C) \land i(C)) \Rightarrow pre(C)\)
\item From Sentence 8 PL-FOL: \(si(C) \Rightarrow \lnot pre(C)\)
\end{itemize}
\subsubsection{Example: Carol can collect loyalty points (Sentence 1. combined with Sentence 6.)}
\label{sec:org01761b3}
\begin{itemize}
\item Sentence 1. is a FOL statement: \(\forall x \: o(x) \Rightarrow l(x)\)
\item Sentence 6. can be interpreted as  a PL statement: \(o(C)\) (This is true in our truth table for the consistent model.)
\item Since Carol is the only object that's a customer in our model, the only instantiation of Sentence 1. that adds value is the following: \(o(C) \Rightarrow l(C)\)
\item Thus, in our model's truth table, \(l(C)\) is true.
\item This is an example of how we will use the provided sentences to prove that a our model is consistent.
\item Note that this example is not critical because there are no other statements that involve \(l(C)\) in PL-FOL
\end{itemize}
\subsubsection{Truth table to show a consistent model.}
\label{sec:orgddc852a}
\begin{itemize}
\item No statement contradicts another statement.
\item S0 referes to "Statement 0" not "Sentence 0"
\item (S2) refers to the combination of decomposition of S2 and the definition of implication to the current statement.
\item Further explanation to why S5-8 are T was deemed unnecessary as it follows from basic PL.
\end{itemize}
\FloatBarrier
\begin{center}
\begin{tabular}{lll}
Ref & PL-FOL & T/F\\
\hline
S0 & \(\lnot s(C)\) & T\\
S1 & \(o(C)\) & T\\
S2 & \(g(C) \land \lnot i(C) \land \lnot s(C) \land \lnot pre(C)\) & T\\
S3 & \(o(C) \Rightarrow l(C)\) & T\\
S1 mp S3 & \(l(C)\) & T\\
S4 & \(o(C) \Rightarrow d(C)\) & T\\
S1 mp S4 & \(d(C)\) & T\\
S5 & \(i(C) \Rightarrow s(C)\) & T (S2)\\
S6 & \(o(C) \Rightarrow (g(C) \lor si(C))\) & T (S2)\\
S7 & \((g(C) \land i(C)) \Rightarrow pre(C)\) & T (S2)\\
S8 & \(si(C) \Rightarrow \lnot pre(C)\) & T (S2)\\
\end{tabular}
\end{center}
\FloatBarriers
\begin{enumerate}
\item Note
\label{sec:org88e7245}
As FOL statements have either true/false values, they could have been used in place of their propositionalized counterparts. I have universally instantiated with C because the language makes more sense and it doesn't change the values for their corresponding T/F.
\end{enumerate}
\subsection{b.}
\label{sec:org7014713}
Assuming SO has been replaced with \(s(C)\), we can prove that there is a contradiction with the following resolution:
Consider S0* to be \(s(C)\)
\subsubsection{Resolution}
\label{sec:org7b50cc4}
\begin{itemize}
\item With decomposition we can move from S2 to \(\lnot s(C)\)
\item Consider S2* to be \(\lnot s(C)\)
\item S2* resolves with S0* to the empty set, thus we have a contradiction and our model is not consistent
\end{itemize}
\section{Question 3}
\label{sec:org872b529}
\begin{itemize}
\item FOR matches solutions
\item MAC incorporates the same tweaks necessary to match FOR solutions but uses slightly different algorithm than TAs.
\item Execution instructions and solutions locations explained:
\end{itemize}
\lstset{language=shell,label= ,caption= ,captionpos=b,numbers=none}
\begin{lstlisting}
./README.pdf
\end{lstlisting}
\section{VCL}
\label{sec:orgcd80901}
Execution was tested on the Linux Lab machine.
\end{document}