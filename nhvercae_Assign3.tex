% Created 2022-04-01 Fri 09:34
% Intended LaTeX compiler: pdflatex
\documentclass[11pt]{article}
\usepackage[utf8]{inputenc}
\usepackage[T1]{fontenc}
\usepackage{graphicx}
\usepackage{longtable}
\usepackage{wrapfig}
\usepackage{rotating}
\usepackage[normalem]{ulem}
\usepackage{amsmath}
\usepackage{amssymb}
\usepackage{capt-of}
\usepackage{hyperref}
\usepackage{minted}
\usepackage{color}
\usepackage{listings}
\author{DESKTOP-OEFS7VE}
\date{\today}
\title{}
\hypersetup{
 pdfauthor={DESKTOP-OEFS7VE},
 pdftitle={},
 pdfkeywords={},
 pdfsubject={},
 pdfcreator={Emacs 27.2 (Org mode 9.5.1)}, 
 pdflang={English}}
\begin{document}

\tableofcontents

\section{Question 1}
\label{sec:org653e72f}
\subsection{a.}
\label{sec:org5f651a4}
\begin{enumerate}
\item \(tea(BlackTea) \land (\forall x \:cheese(x) \Rightarrow \lnot tea(x))\)
\item \(blend(BlackTea, GreenTea)\)
\item \(\forall x,y \: (\lnot tea(x) \land tea(y)) \Rightarrow \lnot blend(y,x)\)
\item \(\exists x \: tea(x) \land unoxidize(x) \land \lnot wilted(x)\)
\item \(\exists x \forall y \: tea(x) \land ((tea(y) \land \lnot unoxidize(y)) \Rightarrow blend(x,y))\)
\end{enumerate}
\subsection{b.}
\label{sec:org3970bfb}
\subsubsection{1.}
\label{sec:org982a5d9}
\begin{itemize}
\item \(tea(BlackTea) \land (cheese(x) \Rightarrow \lnot tea(x))\)
\item \(tea(BlackTea) \land (\lnot cheese(x) \lor \lnot tea(x))\)
\end{itemize}
\subsubsection{2.}
\label{sec:org10a0c9b}
\begin{itemize}
\item \(blend(BlackTea, GreenTea)\)
\end{itemize}
\subsubsection{3.}
\label{sec:org2a3240a}
\begin{itemize}
\item \((\lnot tea(x) \land tea(y)) \Rightarrow \lnot blend(y,x)\)
\item \(\lnot(\lnot tea(x) \land tea(y)) \lor \lnot blend(y,x)\)
\item \((tea(x) \lor \lnot tea(y)) \lor \lnot blend(y,x)\)
\item \(tea(x) \lor \lnot tea(y) \lor \lnot blend(y,x)\)
\end{itemize}
\subsubsection{4.}
\label{sec:org4c12128}
\begin{itemize}
\item \(tea(T1) \land unoxidize(T1) \land \lnot wilted(T1)\)
\end{itemize}
\subsubsection{5.}
\label{sec:org6b89032}
\begin{itemize}
\item \(tea(T2) \land ((tea(y) \land \lnot unoxidize(y)) \Rightarrow blend(T2,y))\)
\item \(tea(T2) \land (\lnot(tea(y) \land \lnot unoxidize(y)) \lor blend(T2,y))\)
\item \(tea(T2) \land ((\lnot tea(y) \lor unoxidize(y)) \lor blend(T2,y))\)
\item \(tea(T2) \land (\lnot tea(y) \lor unoxidize(y) \lor blend(T2,y))\)
\end{itemize}
\subsection{c.}
\label{sec:org9b00939}
\section{Question 2}
\label{sec:org8440369}
Note that I use PL-FOL to refer to propositionalization.
\subsection{a.}
\label{sec:orgff5b2dc}
\subsubsection{Provided Sentences in FOL}
\label{sec:org1c5b150}
\begin{itemize}
\item Sentence 1: \(\forall x \: o(x) \Rightarrow l(x)\)
\item Sentence 2: \(\forall x \: o(x) \Rightarrow d(x)\)
\item Sentence 3: \(\forall x \: (o(x) \land i(x)) \Rightarrow s(x)\)
\item Sentence 3 simplified: \(\forall x \: i(x) \Rightarrow s(x)\)
\item The remaining sentences will take this simplified approach. It is a safe assumption that our model does not have any assignments that would complicate these simplifications.
\item From Sentence 5: \(\forall x \: o(x) \Rightarrow (g(x) \lor si(x))\)
\item From Sentence 8: \(\forall x \: (g(x) \land i(x)) \Rightarrow pre(x)\)
\item From Sentence 8: \(\forall x \: si(x) \Rightarrow \lnot pre(x)\)
\end{itemize}
\subsubsection{Provided sentences that can be interpreted in terms of PL-FOL}
\label{sec:orgbc88176}
\begin{itemize}
\item Sentence 4: \(\lnot s(C)\)
\item Sentence 6: \(o(C)\)
\item Sentence 7: \(g(C) \land \lnot i(C) \land \lnot s(C) \land \lnot pre(C)\)
\end{itemize}
\subsubsection{FOL Sentences instantiated into PL (Carol is the only customer, so she is the only instatiation that matters)}
\label{sec:orgc9617fa}
\begin{itemize}
\item Sentence 1 PL-FOL: \(o(C) \Rightarrow l(C)\)
\item Sentence 2 PL-FOL: \(o(C) \Rightarrow d(C)\)
\item Sentence 3 PL-FOL: \(i(C) \Rightarrow s(C)\)
\item Sentence 5 PL-FOL: \(o(C) \Rightarrow (g(C) \lor si(C))\)
\item From Sentence 8 PL-FOL: \((g(C) \land i(C)) \Rightarrow pre(C)\)
\item From Sentence 8 PL-FOL: \(si(C) \Rightarrow \lnot pre(C)\)
\end{itemize}
\subsubsection{Example: Carol can collect loyalty points (Sentence 1. combined with Sentence 6.)}
\label{sec:org4216395}
\begin{itemize}
\item Sentence 1. is a FOL statement: \(\forall x \: o(x) \Rightarrow l(x)\)
\item Sentence 6. can be interpreted as  a PL statement: \(o(C)\) (This is true in our truth table for the consistent model.)
\item Since Carol is the only object that's a customer in our model, the only instantiation of Sentence 1. that adds value is the following: \(o(C) \Rightarrow l(C)\)
\item Thus, in our model's truth table, \(l(C)\) is true.
\item This is an example of how we will use the provided sentences to prove that a our model is consistent.
\item Note that this example is not critical because there are no other statements that involve \(l(C)\) in PL-FOL
\end{itemize}
\subsubsection{Truth table to show a consistent model.}
\label{sec:orgedaf441}
No statement contradicts another statement.
\FloatBarrier
\begin{center}
\begin{tabular}{lll}
Ref & PL-FOL & T/F\\
\hline
S0 & \(\lnot s(C)\) & T\\
S1 & \(o(C)\) & T\\
S2 & \(g(C) \land \lnot i(C) \land \lnot s(C) \land \lnot pre(C)\) & T\\
S3 & \(o(C) \Rightarrow l(C)\) & T\\
S1 mp S3 & \(l(C)\) & T\\
S4 & \(o(C) \Rightarrow d(C)\) & T\\
S1 mp S4 & $\backslash$(d(C)) & T\\
S5 & \(i(C) \Rightarrow s(C)\) & T (S2)\\
S6 & \(o(C) \Rightarrow (g(C) \lor si(C))\) & T (S2)\\
S7 & \((g(C) \land i(C)) \Rightarrow pre(C)\) & T (S2)\\
S8 & \(si(C) \Rightarrow \lnot pre(C)\) & T (S2)\\
\end{tabular}
\end{center}
\FloatBarriers
\end{document}